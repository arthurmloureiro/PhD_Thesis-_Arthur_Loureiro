% !TEX root = ../thesis-example.tex
%
\pdfbookmark[0]{Abstract}{Abstract}
\chapter*{Abstract}
\label{sec:abstract}
\vspace*{-10mm}

The last decade of the 20th Century produced two Noble Prize winning, revolutionary, discoveries that changed both standard models of particle physics and cosmology. The analysis of Supernovae Type Ia light-curves showed that the Universe is going through an accelerated expansion phase -- leading to the LCDM paradigm. Meanwhile, solar and atmospheric neutrino experiments demonstrated that neutrinos oscillate between their three leptonian flavours -- implying they have non-vanishing mass eigen-states. Now, almost 20 years later, cosmology has shifted towards being a precise science with an increasing amount of data. Uncertainties in cosmological parameters' measurement have been reduced to a percent level. Recent studies presented in this work seem to also confirm the flat-LCDM paradigm. In this work, it is demonstrated that an angular power spectra analysis of spectroscopic galaxy surveys can overcome most of these difficulties while still extracting competitive and powerful constraints -- when compared to results from most large cosmology collaborations. This important result breaks the paradigm for galaxy survey analysis, probing the evolution of structure formation with more detail than the standard 3D analysis. This work advocates for tomographic analysis to be used as a unified framework for all galaxy surveys. Such framework allows to optimally combine different types of cosmological observations, taking into account cross-correlations between a variety of probes and samples while allowing for modelling and mitigation of errors in a robust way. Using this approach, it is demonstrated that current cosmological data can be combined with neutrino oscillation experiments to yield robust constraints on the upper bounds of the sum of neutrino masses and the lightest neutrino species. Finally, a generic Bayesian angular power spectra estimator for spin-0 and spin-2 fields is presented with applications to Euclid-like galaxy clustering simulations as a solution to reliably probing large scales for future studies related to primordial non-gaussianity .
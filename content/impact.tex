\pdfbookmark[0]{Impact Statement}{Impact Statement}
\chapter*{Impact Statement}
\label{sec:impact}
\vspace*{-10mm}

In this work, I demonstrate the full potential of performing an analysis using spherical harmonics to obtain competitive cosmological information from redshift galaxy surveys. Demonstrating that this approach is as powerful, if not more powerful than, the standard 3D analysis performed in spectroscopic surveys, shows that this is the perfect unified framework to be used when combining cosmological probes in the future. Even though the cosmological analysis using the angular power spectra of BOSS galaxies did not yield new cosmological insights, I was able to measure the equation-of-state of dark energy with the same accuracy as other state-of-the-art cosmology collaborations. This is a definitive demonstration of this approach's power.

\qquad Beyond that, I have demonstrated that cosmological probes can be combined with Particle Physics constraints from neutrino oscillation experiments in order to obtain the first ever upper bound on the mass of the lightest neutrino species -- independently of neutrinos being Dirac or Majorana. This, as far as I am aware, was never done before with real data. These results open a whole new door for collaborations between particle physicists and cosmologists -- allowing galaxy surveys to break the Standard Model of Cosmology while trying to fix the Standard Model of Particle Physics by shining a light into neutrino masses from a new and different perspective.

\qquad Results related to the lightest neutrino mass from combined cosmological probes and oscillation experiment constraints were submitted to Physical Review Letters. Once accepted, these findings will be part of a press-release to inform the general public outside the scientific community. I have already performed talks in schools about `Hunting Ghosts with Galaxies' to communicate the general public about neutrino physics, modern cosmology, and the importance of galaxy surveys for the future of science. The impact of this results are clear as day and can change future research beyond just the way we probe neutrinos in cosmology.

\qquad Finally, on the last chapter, I present application of a Bayesian angular power spectra estimator for galaxy surveys. The final interest here is to provide the Euclid Collaboration (to which I am a member) with more precise and accurate option to obtain galaxy clustering measurements. This method helps obtaining precise information on the large scale end of the power spectrum, where information related to new physics could be hiding -- making this work crucial for stepping towards a new paradigm in cosmology.